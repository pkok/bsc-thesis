\section{Theory}
\label{theory}

This section gives a short introduction to a mathematical framework for robot movement.  Before this can be done, some common concepts in robotics need to be introduced.  A basic understanding of linear algebra is assumed.


\subsection{Robotics}
\label{theory/robotics}
Robots are physical agents that performs tasks by manipulating the physical world \cite[p.901]{Russel2003}.  A robot's body consists of \emph{links}\index{link} which are connected with each other through \emph{joints}\index{joint}.  For the discussion in this document it is assumed that all links are \emph{rigid bodies}\index{rigid body}; physical objects that cannot be deformed.  A link's position can change with respect to another link through change in its joint.  It can either revolve around a point, or it can be moved some distance.  Joints that enable the first type of change are called \emph{revolute}\index{joint!revolute}, the second class is called a \emph{prismatic joint}\index{joint!prismatic}.  We count 1 \emph{degree of freedom}\index{degree of freedom} (DOF\index{DOF}) for each transformation a joint can exercise on a link.  Joints with a DOF greater than 1 are said to be composite, and can be represented by multiple joints that are connected to each other by links with no length.  

Links are thus connected to other links by joints.  When seen as a single object, an uninterrupted chain of links and joints is called a \emph{kinematic chain}\index{kinematic chain}.  Kinematic chains can be \emph{closed}\index{kinematic chain!closed}; it can contain a loop, where a link is connected to another link that is already in the chain.  When there are no loops, the kinematic chain is said to be \emph{open}\index{kinematic chain!open}.

Because of its links and joints, the robot can affect the world.  The parts of its body that do so, are called \emph{effectors}\index{effector}.  The parts at the very end of an (open) kinematic chain are called \emph{end effectors}\index{effector!end}.

\comment{Links hebben een co\"ordinatenstelsel, niet joints toch?}

For some problems it is more convenient to know where an object is relative to a link, than knowing where the link and the object are relative to the origin of the world coordinate system.  For this convenience, one may define a \emph{frame}\index{frame}, or basis, $\C{B}_i$ for each link $l_i$.  The basis is usually orthogonal.  The origin of the basis is defined relative to its corresponding link.  We will use $[\V{p}]_{\C{B}}$ to denote a vector that represents a point $p$ in the basis $\C{B}$.  As a shorthand for $[\V{p}]_\C{W}$, a vector representing a point $p$ in the basis of the world, we write $\V{p}$.

As rigid bodies cannot be deformed, they can only undergo operations that are \emph{isometric}\index{isometry}.  A transformation $\T{A}$ is said to be isometric when it preserves the distance between vectors: \[ \forall \V{x}, \V{y} \in \mathbb{R}^n: \V{x} \cdot \V{y} = \T{A}\V{x} \cdot \T{A}\V{y}.\]  Each isometry in a Euclidean space can be decomposed by a rotation (or a special orthogonal transformation), a translation and a reflection.  As a physical object cannot reflect its body in the real world, we will not further discuss those isometries that reflect points in a certain plane.

\TODO{Decompositiestelling kort behandelen}

\TODO{Transformatie in kleine stapjes toepassen $\to$ geen reflectie!}

\TODO{Stelling ``2 spiegelingen maken 1 rotatie'' bespreken}

\TODO{Denavit-Hartenberg bespreken!}

\comment{Verhaal verduidelijken aan de hand van een plaatje?}


\subsection{Forward and inverse kinematics}
\label{theory/kinematics}

Forward and inverse kinematics form a framework for working with robot joint configurations and their positions.  This discussion assumes all kinematic chains to be open.

Forward kinematics is used to compute the position and orientation of a link with respect to some basis $\C{B}$, given the configuration of the joints in the kinematic chain.

Each link $l_i$ in the kinematic chain has its own coordinate system $\C{B}_i$. One can express a position  $[\V{p}]_{\C{B}_i}$ given in a certain coordinate system $\C{B}_i$, in a different coordinate system $\C{B}_j$, when the isometric transformation between $\C{B}_i$ and $\C{B}_j$ is known.  As shown before, a isometric transformation can be decomposed into a rotation, translation and reflection.  
