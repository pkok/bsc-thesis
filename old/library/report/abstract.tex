\thispagestyle{plain}

\vspace*{1.5cm}

\phantomsection
\addcontentsline{toc}{chapter}{Abstract}
\begin{center}
\textbf{\color{blueish}Abstract}
\end{center}
\vspace*{-0.4cm}
In this thesis I present a method to keep a point on a robotic non-prismatic
limb in a plane relative to the coordinate system of another non-prismatic
limb, while both limbs can move.  The position of the limbs is tracked solely
by its shaft encoders.\comment{Zijn dat shaft encoders?}  As an example of
practical use of this method, I have implemented a method of generating
force feedback to an operator when controling a robot arm.  It generates the
force feedback when a laser range finder on the robot's head detects an
obstacle in front of the robot hand.

\vspace*{1.5cm}

\phantomsection
\addcontentsline{toc}{chapter}{Acknowledgement}
\begin{center}
\textbf{\color{blueish}Acknowledgement}
\end{center}
\vspace*{-0.4cm}
Of course I have people to thank.


\vfill
\noindent Universiteit van Amsterdam\\
Instituut voor Informatica\\
Science Park 904\\
1098 XG Amsterdam
\begin{center}
\includegraphics{uva_logo}
\end{center}
